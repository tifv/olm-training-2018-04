% $date: 2018-04-13
% $timetable:
%   g8:
%     2018-04-13:
%       2:

% $caption:
%   Частично упорядоченные множества, цепи, антицепи
%   и теорема Дилуорса

\worksheet*{Частично упорядоченные множества, цепи, антицепи
    и~теорема Дилуорса}

% $authors:
% - Андрей Юрьевич Кушнир

\begingroup
    \def\arrow{\mathord{\rightarrow}}%
    \def\abs#1{\lvert #1 \rvert}%

\claim{Определения}
\emph{Частично упорядоченным множеством} называют ориентированный граф без
петель и~кратных стрелок, для которого выполнено свойство
\emph{транзитивности}:
для любых трех вершин $u$, $v$, $w$ таких, что $u \arrow v$ и~$v \arrow w$,
верно $u \arrow w$.

В~частично упорядоченном множестве определены понятия цепи и~антицепи.
\emph{Цепь}~---~последовательность вершин вида
$u_{1} \arrow u_{2} \arrow \ldots \arrow u_{k}$.
\emph{Антицепь}~--- множество вершин, никакие две из~которых не~соединены
стрелкой.

\begin{problems}

%Определение
%Частично упорядоченное множество~--- это множество, для некоторых элементов
%которого задано отношение <<{<}>>, удовлетворяющее соотношению антирефлексивности
%(не~$a < a$) и~транзитивности
%($a < b \text{ и } b < c \enspace\Rightarrow\enspace a < c$).
%Цепь~--- это последовательность элементов $m_{1}, m_{2}, \ldots, m_{k}$, такая
%что $m_1 < m_2 < \ldots < m_k$.
%Антицепь~--- это множество элементов $m_{1}, m_{2}, \ldots, m_{k}$, таких, что
%никакие два из~них не~сравнимы друг с~другом.

\item
Докажите, что в~частично упорядоченном множестве размер максимальной цепи равен
минимальному количеству антицепей, на~которые оно разбивается.

\item
Докажите, что в~частично упорядоченном множестве количество элементов не~больше
размера максимальной антицепи, умноженного на~размер максимальной цепи.

\item
На~cannibal-party собрались $100$ людоедов.
По~окончании вечеринки выяснилось, что среди любых $10$ людоедов найдутся два,
один из~которых съел другого.
Докажите, что найдется цепочка из~$12$ людоедов $C_1, C_2, \ldots, C_{12}$
такая, что при $i < j$ людоед $C_i$ съел людоеда $C_j$.

%\item
%Петя вырезал из~бумаги $2018$ различных прямоугольников с~целыми сторонами,
%не~превышающими $1000$.
%Докажите, что среди них можно выбрать пять таких, что каждый следующий строго
%больше предыдущего.

\item
Для данных натуральных $r$ и~$s$ покажите, что любая последовательность
различных чисел длины $(r - 1) (s - 1) + 1$ содержит монотонно возрастающую
подпоследовательность длины $r$ или монотонно убывающую длины $s$.

\item
Построим по~частично упорядоченному множеству~$M$ двудольный граф~$G$ следующим
образом.
В~левую долю $G$ поместим копии элементов $M$.
В~правую долю $G$ тоже поместим копии элементов $M$.
Вершина~$u$ левой доли соединена с~вершиной~$v$ правой доли тогда и~только
тогда, когда в~$M$ верно $u \arrow v$.
\\
\subproblem
Докажите, что размер минимального разбиения $M$ на~цепи равен
\[
    \abs{M} - (\text{размер максимального паросочетания в~$G$})
\, . \]
\par
\subproblem
Докажите, что размер максимальной антицепи в~множестве~$M$ равен
\[
    \abs{M} - (\text{размер минимального вершинного покрытия ребер в~$G$})
\, . \]
\par
\subproblem
\claim{Теорема Дилуорса}
Докажите, что в~частично упорядоченном множестве размер максимальной антицепи
равен минимальному количеству цепей, на~которые оно разбивается.

%\item
%На~олимпиаде школьникам был предложен вариант из~$n$ задач.
%Оказалось, что для любых двух школьников нашлась задача, которую решил первый
%и~не~решил второй, и~нашлась задача, которую решил второй и~не~решил первый.
%Какое максимальное количество школьников могло участвовать на~олимпиаде?

%\item
%Из~чисел от~$1$ до~$100$ выбрали некоторое подмножество $M$ таким образом, что
%никакие два числа из~$M$ не~делятся одно на~другое.
%Какой максимальный размер может быть у~множества $M$?

%\item
%Компания из~$n$ музыкантов дает по~вечерам концерты, при этом каждый вечер
%часть музыкантов играет на~сцене, а~остальные~--- слушатели.
%Какое минимальное число вечеров понадобится для того, чтобы каждый хотя~бы
%по~разу послушал каждого?

%\item
%Рассмотрим всевозможные расстановки чисел от~$1$ до~$n$ в~ряд такие, в~которых
%не~существует трех не~обязательно подряд идущих убывающих элементов.
%Докажите, что их количество не~превосходит $4^n$.
%% Каталан, ты?

%\item
%В~графе $200k$ ребер и~$200$ вершин, ребра занумерованы числами от~$1$
%до~$200k$.
%Докажите, что найдется путь (не~обязательно простой) длины $k$, в~котором числа
%на~ребрах возрастают при движении от~одного конца пути к~другому.

\end{problems}

\endgroup % \def\arrow \def\abs

