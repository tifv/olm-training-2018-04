% $date: 2018-04-05
% $timetable:
%   gXs:
%     2018-04-05:
%       1:

\worksheet*{Клетки}

% $authors:
% - Фёдор Владимирович Петров

\begin{problems}

\item
Боковая поверхность параллелепипеда с~основанием $a\times b$ и~высотой~$c$
($a, b, c \in \mathbb{N}$) оклеена без наложений и~пропусков прямоугольниками
по~клеточкам, каждый прямоугольник имеет четную площадь и~может перегибаться
по~ребру (хоть по~четырем).
Докажите, что если $c$ нечетно, то~число способов оклейки четно.
% Р-97-9.3

\item
В~клетках таблицы $10 \times 10$ расставлены числа $1$, $2$, \ldots, $100$ так,
что сумма любых двух соседних по~стороне не~превосходит $S$.
Найдите наименьшее возможное значение $S$.
% Р-97-9.8

\item
В~прямоугольную таблицу $m \times n$, где $m$ и~$n$ нечетны, уложены домино
размера $2 \times 1$ так, что остался непокрыт только квадрат $1 \times 1$
в~углу коробки.
Если доминошка прилегает к~дырке короткой стороной, ее разрешается сдвинуть
вдоль себя на~одну клетку, закрыв дырку (при этом открывается новая дырка).
Докажите, что с~помощью таких передвижений можно перегнать дырку в~любой другой
угол.
% Р-97-11.8

\item
В~каждую клетку таблицы $(2^{n} - 1) \times (2^{n} - 1)$ ставится одно из~чисел
$+1$ или $-1$.
Расстановку чисел назовем \emph{удачной,} если каждое число равно произведению
всех соседних с~ним по~стороне.
Найдите число удачных расстановок.
% Р-98-10.8

\item
В~квадрате $n \times n$ клеток бесконечной шахматной доски расположены
$n^2$~фишек в~каждой клетке.
Ходом называется перепрыгивание любой фишкой через соседнюю по~стороне фишку,
непосредственно за~которой следует свободная клетка.
При этом фишка, через которую перепрыгнули, с~доски снимается.
Докажите, что позиция, в~которой дальнейшие ходы невозможны, возникнет
не~ранее, чем через $[\frac{n^2}{3}]$ ходов.
% Р-99-10.4, 11.4

\item
Можно~ли все клетки какой-нибудь прямоугольной таблицы окрасить в~белый
и~черный цвета так, чтобы белых и~черных клеток было поровну, а~в~каждой строке
и~в~каждом столбце было более 3/4 клеток одного цвета?
% Союз-81-10.4

\item
Клетки квадрата $100 \times 100$ покрашены в~четыре цвета, причем в~каждой
строке и~в~каждом столбце клеток всех четырех цветов поровну.
Докажите, что центры некоторых четырех разноцветных клеток образуют
прямоугольник со~сторонами, параллельными сторонам квадрата.
% Р-00-11.8

\item
Прямоугольник разбит на~доминошки.
Докажите, что его клетки можно раскрасить в~два цвета так, чтобы любая
доминошка в~данном разбиении содержала клетки разных цветов, но~в~любом другом
разбиении нашлась~бы доминошка, содержащая две клетки одного цвета.
% отбор-95-9.6

\item
Квадрат $8 \times 8$ выложен из~единичных кубиков.
Можно~ли их уложить в~виде квадрата $4 \times 4 \times 4$ так, чтобы соседние
кубики остались соседними?
% 87.24 Фомин

\item
Имеется таблица $100 \times 100$, все клетки которой покрашены в~три цвета.
Разрешается перекрашивать любой квадратик $2 \times 2$ в~тот цвет, который
в~нем преобладает, а~если такого нет, то~в~тот цвет, которого нет в~квадратике.
Докажите, что весь квадратик можно перекрасить в~один цвет.
% 76.35

\end{problems}

