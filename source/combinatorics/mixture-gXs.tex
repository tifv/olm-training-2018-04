% $date: 2018-04-07
% $timetable:
%   gXs:
%     2018-04-07:
%       1:

\worksheet*{Комби}

% $authors:
% - Фёдор Владимирович Петров

\begin{problems}

\item
Клетки шахматной доски $n \times n$, $n > 2$, раскрашены $n^2 / 2$ красками
так, что каждой краской покрашено ровно две клетки.
Докажите, что можно так расставить на~доске $n$ ладей, чтобы они стояли
на~клетках $n$ различных цветов и~не~били друг друга.

\item
Множество~$X$ разбито на~попарно непересекающиеся подмножества
$A_{1}$, $A_{2}$, \ldots, $A_{n}$, а~также разбито на~попарно непересекающиеся
подмножества $B_{1}$, $B_{2}$, \ldots, $B_{n}$.
Известно, что объединение любых двух непересекающихся подмножеств
$A_{i}$, $B_{j}$ ($1 \leq i, j \leq n$) содержит не~менее $n$~элементов.
Докажите, что число элементов множества~$X$ не~меньше $n^2/2$.

\item
В~каждой клетке шахматной доски написано положительное число так, что в~каждой
горизонтали сумма чисел равна~1.
Известно, что при любой расстановке восьми не~бьющих друг друга ладей на~доске
произведение чисел под ними не~больше произведения чисел на~главной диагонали.
Докажите, что сумма чисел на~главной диагонали не~меньше~1.

\item
Докажите, что в~$n$-вершинном графе без 4-циклов не~более
$\frac{1}{2} n \left( \frac{1}{2} + \sqrt{n - \frac{3}{4}} \right)$
ребер.

\item
В~стране 1998 городов и~из~каждого осуществляются беспосадочные перелеты в~три
других города.
Известно, что из~любого города, сделав несколько пересадок, можно долететь
до~любого другого.
Министерство Безопасности хочет объявить закрытыми 200 городов, никакие два
из~которых не~соединены авиалинией.
Докажите, что это можно сделать так, чтобы можно было долететь из~любого
незакрытого города в~любой другой, не~делая пересадок в~закрытых городах.

\item
\subproblem
В~городе Незнакомске живут $3n$~человек, причем любые двое имеют общего
знакомого.
Докажите, что можно указать $n$~человек таким образом, чтобы каждый
из~остальных был знаком хотя~бы с~одним человеком из~этих $n$.
\\
\subproblem
В~городе Незнакомске миллион жителей, причем любые двое имеют общего знакомого.
Докажите, что можно указать 5000 человек таким образом, чтобы каждый
из~остальных был знаком хотя~бы с~одним человеком из~этих 5000.

\item
В~прямоугольной таблице строк больше, чем столбцов, а~в~некоторых клетках стоят
звездочки, причем в~каждой строке есть хотя~бы одна.
Докажите, что есть звездочка, в~строке которой стоит меньше звездочек, чем в~ее
столбце.

\end{problems}

%\item
%На~балу встретились юноши и~девушки, причем их было поровну.
%Каждый юноша знаком с~7 девушками.
%Известно, что они могут танцевать вальс так, что танцующие в~паре будут знакомы
%между собой.
%Докажите, что они могут таким образом разбиваться на~пары не~менее, чем 5000
%способами.

%\item
%В~стране 2000 городов, каждые два из~которых соединены дорогой.
%Строительные организации представили все возможные проекты введения
%одностороннего движения на~всех дорогах.
%Министерство транспорта отвергло все проекты, не~обеспечивавшие возможности
%добраться из~любого города в~любой другой.
%Докажите, что все~же осталось более половины проектов.

%\item
%Боковая поверхность параллелепипеда с~основанием $a\times b$ и~высотой $c$
%($a,\,b,\,c\in \mathbb{N}$) оклеена без наложений и~пропусков прямоугольниками
%по~клеточкам, каждый прямоугольник имеет четную площадь и~может перегибаться
%по~ребру (хоть по~четырем).
%Докажите, что если $c$ нечетно, то~число способов оклейки четно.
%% Р-97-9.3

%\item
%В~клетках таблицы $10\times 10$ расставлены числа $1,\,2,\,\dots,100$ по~разу
%так, что сумма чисел в~любых двух соседних по~стороне клетках не~превосходит
%$S$.
%Найдите наименьшее возможное значение $S$.
%% Р-97-9.8

%\item
%В~прямоугольную таблицу $m\times n$, где $m$ и~$n$ нечетны, уложены домино
%размера $2\times 1$ так, что остался непокрыт только квадрат $1\times 1$ в~углу
%коробки.
%Если доминошка прилегает к~дырке короткой стороной, ее разрешается сдвинуть
%вдоль себя на~одну клетку, закрыв дырку (при этом открывается новая дырка).
%Докажите, что с~помощью таких передвижений можно перегнать дырку в~любой другой
%угол.
%% Р-97-11.8

%\item
%В~каждую клетку таблицы $(2^n-1)\times (2^n-1)$ ставится одно из~чисел $+1$ или
%$-1$.
%Расстановку чисел назовем удачной, если каждое число равно произведению всех
%соседних с~ним по~стороне.
%Найдите число удачных расстановок.
%% Р-98-10.8

%\item
%В~квадрате $n\times n$ клеток бесконечной шахматной доски расположены $n^2$
%фишек в~каждой клетке.
%Ходом называется перепрыгивание любой фишкой через соседнюю по~стороне фишку,
%непосредственно за~которой следует свободная клетка.
%При этом фишка, через которую перепрыгнули, с~доски снимается.
%Докажите, что позиция, в~которой дальнейшие ходы невозможны, возникнет
%не~ранее, чем через $[{n^2\over 3}]$ ходов.
%% Р-99-10.4,11.4

%\item
%Можно~ли все клетки какой-нибудь прямоугольной таблицы окрасить в~белый
%и~черный цвета так, чтобы белых и~черных клеток было поровну, а~в~каждой строке
%и~в~каждом столбце было более 3/4 клеток одного цвета?
%% Союз-81-10.4

%\item
%Клетки квадрата $100\times 100$ покрашены в~четыре цвета, причем в~каждой
%строке и~в~каждом столбце клеток всех четырех цветов поровну.
%Докажите, что центры некоторых четырех разноцветных клеток образуют
%прямоугольник со~сторонами, параллельными сторонам квадрата.
%%Р-00-11.8

%\item
%Прямоугольник разбит на~доминошки.
%Докажите, что его клетки можно раскрасить в~два цвета так, чтобы любая
%доминошка в~данном разбиении содержала клетки разных цветов, но~в~любом другом
%разбиении нашлась~бы доминошка, содержащая две клетки одного цвета.
%% отбор-95-9.6

%\item
%Квадрат $8\times 8$ выложен из~единичных кубиков.
%Можно~ли их уложить в~виде куба $4\times 4\times 4$ так, чтобы соседние кубики
%остались соседними?
%% 87.24 Фомин

%\item
%Имеется таблица $100\times 100$, все клетки которой покрашены в~три цвета.
%Разрешается перекрашивать любой квадратик $2\times 2$ в~тот цвет, который в~нем
%преобладает, а~если такого нет, то~в~тот цвет, которого нет в~квадратике.
%Докажите, что весь квадрат можно перекрасить в~один цвет.
%% 76.35

