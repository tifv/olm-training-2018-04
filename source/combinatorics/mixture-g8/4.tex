% $date: 2018-04-06
% $timetable:
%   g8:
%     2018-04-06:
%       3:

\worksheet*{Комбинаторика. Серия 4}

% $authors:
% - Андрей Юрьевич Кушнир

\begin{problems}

\item
Даны $n$~точек общего положения, два игрока по~очереди соединяют из~стрелками.
Нельзя соединять стрелкой уже соединенную пару точек.
Цель второго игрока состоит в~том, чтобы в~конце игры, когда все пары точек
будут соединены, сумма полученных векторов оказалась равной $\vec{0}$.
Сможет~ли первый игрок ему помешать?

\item
Можно~ли за~круглым столом рассадить $12$~человек и~поставить $28$~бутылок
на~стол с~условием, чтобы между любыми двумя людьми стояла
бутылка?
% Лемма Холла посложнее

\item
Даны $n$ мальчиков и~$2n - 1$ конфет.
Докажите, что можно дать каждому мальчику по~конфете так, чтобы мальчику,
которому не~нравится его конфета, не~нравились и~конфеты остальных мальчиков
(чтобы не~создавать предпосылок для драки).

\item
Есть клетчатая доска $2015 \times 2015$.
Дима ставит в~$k$~клеток по~детектору.
Затем Коля располагает на~доске клетчатый корабль в~форме квадрата
$1500 \times 1500$.
Детектор в~клетке сообщает Диме, накрыта эта клетка кораблем или нет.
При каком наименьшем $k$ Дима может расположить детекторы так, чтобы
гарантированно восстановить расположение корабля?

\item
В~некоторых клетках доски $100 \times 100$ стоят фишки.
Оказалось, что в~объединении любой строки и~любого столбца, за~исключением их
общей клетки, стоят хотя~бы две фишки.
Какое наименьшее число фишек могло быть на~доске?

\item
В~детской выездной школе после отбоя вожатый пытается поймать нарушителя
спокойствия.
Корпус лагеря состоит из~$n$ комнат, расположенных в~ряд.
Каждую минуту вожатый проверяет одну из~комнат на~предмет наличия в~ней
нарушителя.
После того, как вожатый покидает комнату, нарушитель мгновенно через окно
перебирается в~одну из~соседних комнат (нельзя оставаться на~месте).
Ни~начальное положение, ни~перемещения нарушителя вожатому не~известны.
За~какое минимальное время вожатый сможет гарантированно поймать нарушителя?

\end{problems}

