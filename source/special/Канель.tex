% $date: 2018-04-08

\worksheet*{Внешний биллиард вокруг правильного многоугольника}

% $authors:
% - Алексей Яковлевич Канель-Белов

Рассмотрим многоугольник $M$.
Из точки $A$ на плоскости проведем касательную (т.\,е. опорную прямую) к $M$
и отразим $A$ относительно точки касания;
такое преобразование называется преобразованием внешнего биллиарда.

Если применять такую операцию к точке многократно, то точка может оказаться
\emph{периодической}, т.\,е. в какой-то момент вернуться в себя~--- а может
и не вернуться, т.\,е. оказаться \emph{апериодической}.
Планируется рассказать о том, как устроены периодически точки, какие алгоритмы
могут быть полезны для компьютерных экспериментов, и почему компьютер
оказывается практически необходимым для полноценного исследования.

