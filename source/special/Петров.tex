% $date: 2018-04-08

\worksheet*{Многомерная игра сет}

% $authors:
% - Фёдор Владимирович Петров

Есть $n$ признаков предмета, принимающих по три возможных значения
(например, предмет может быть круглым, квадратным либо треугольным; при том
синим, красным или белым; а также солёным, сладким или кислым).
Сет~--- это три предмета, которые по каждому признаку либо все различны, либо все
совпадают (например, все треугольные, при этом один синий сладкий, второй
красный кислый, третий белый солёный.)
При каком наименьшем количестве предметов заведомо найдётся сет?
Недавняя сенсация~--- оценка $2{,}76^{n}$
(основная идея принадлежит Э. Круту, В. Льву и П. Паху, её модификация для игры
сет~--- Дж. Элленбергу и Д. Гийсвийту)
доказывается неожиданно просто, и я надеюсь изложить доказательство полностью.
Знакомство с основами линейной алгебры желательно, но не обязательно.

