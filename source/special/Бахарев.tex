% $date: 2018-04-08

\worksheet*{Вариационный принцип в~математике}

% $authors:
% - Фёдор Львович Бахарев

В~трактате Иоганна Кеплера <<Стереометрия винных бочек>> написано:
<<По~обе стороны от~места наибольшего значения убывание вначале
нечувствительно>>.
Мы переформулируем это простое правило, <<вариационный принцип>>, чуть более
аккуратно и~попытаемся понять, как его использовать для решения хорошо
известных задач.
Мы докажем с~помощью него ряд классических неравенств, решим несколько задач
из~геометрии на~максимум и~минимум.
 
Предполагается рассказать и~о~более сложных задачах, которые решаются
вариационным методом: задача о~цепной линии, о~брахистохроне, о~мыльной пленке.
Решить мы их, скорее всего, не~сможем, но~зато, может быть, научимся ездить
на~велосипеде с~квадратными колесами.
 
Первая половина доклада, я надеюсь, будет доступна всем.

