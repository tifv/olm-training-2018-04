% $date: 2018-04-02
% $timetable:
%   g11:
%     2018-04-02:
%       2:

\worksheet*{Первый разнобой}

% $authors:
% - Артемий Алексеевич Соколов

\begin{problems}

%\item
%В~остроугольном треугольнике проведены высоты $AH_{A}$, $BH_{B}$ и~$CH_{C}$.
%Докажите, что треугольник, образованный ортоцентрами
%треугольников $AH_{B}H_{B}$, $BH_{A}H_{C}$ и~$CH_{A}H_{B}$, равен
%треугольнику $H_{A}H_{B}H_{C}$.

\item
Через центр вписанной окружности $I$ треугольника $ABC$ провели прямую,
пересекающую стороны $AB$ и~$BC$ треугольника в~точках $M$ и~$N$
соответственно, так, что треугольник $MBN$ остроугольный.
Описанные окружности треугольников $IMA$ и~$INC$ вторично пересекают
отрезок~$AC$ в~точках $P$ и~$Q$.
Докажите, что $AC = AM + PQ + CN$.

\item
Неравнобедренный треугольник $ABC$ вписан в~окружность~$\omega$.
Касательная к~этой окружности в~точке~$C$ пересекает прямую~$AB$ в~точке~$D$.
Пусть $I$~--- центр вписанной окружности, треугольника $ABC$.
Прямые $AI$ и~$BI$ пересекают биссектрису угла $CDB$ в~точках $Q$ и~$P$
соответственно.
Пусть $M$~--- середина отрезка~$PQ$.
Докажите, что прямая~$MI$ проходит через середину дуги~$ACB$
окружности~$\omega$.

\item
Через центр~$O$ описанной окружности остроугольного треугольника $ABC$
проведена прямая, пересекающая стороны $AB$, $AC$ в~точках $X$, $Y$.
Эти точки отразили относительно середин сторон, на~которых они лежат и~получили
точки $X'$, $Y'$.
Докажите, что $\angle X'HY' = \angle BAC$, где $H$~--- ортоцентр
треугольника $ABC$.

%\item
%В~остроугольном треугольнике $ABC$ отметили ортоцентр~$H$ и~середину~$M$
%стороны~$AC$.
%Через точку~$H$ провели прямую, перпендикулярную прямой~$MH$, которая
%пересекает стороны $AB$ и~$BC$ в~точках $P$ и~$Q$ соответственно.
%Докажите, что $PH = HQ$.

\item
Пусть $H$ и~$O$~--- ортоцентр и~центр описанной окружности треугольника $ABC$.
Окружность, описанная около треугольника $AOH$, пересекает серединный
перпендикуляр к~$BC$ в~точке $A_{1}$.
Аналогично определяются точки $B_{1}$ и~$C_{1}$.
Докажите, что прямые $AA_{1}$, $BB_{1}$ и~$CC_{1}$ пересекаются в~одной точке.

%\item
%Периметр треугольника $ABC$ равен $4$.
%На~лучах $AB$ и~$AC$ отмечены точки $X$ и~$Y$ так, что $AX = AY = 1$.
%Отрезки $BC$ и~$XY$ пересекаются в~точке~$M$.
%Докажите, что периметр одного из~треугольников $ABM$ и~$ACM$ равен 2.

\item
Точка~$X$ вне треугольника $ABC$ такова, что $A$ лежит внутри
треугольника $BXC$.
При этом $2 \angle XBA = \angle ACB$, $2 \angle XCA = \angle ABC$.
Докажите, что центры описанной и~вневписанной со~стороны~$BC$ окружностей
треугольника $ABC$ и~точка~$X$ лежат на~одной прямой.

\item
Дан неравнобедренный остроугольный треугольник $ABC$.
Точки $I$ и $I_{A}$~--- центры вписанной и~вневписанной окружностей
треугольника $ABC$ напротив вершины $A$ соответственно.
Вписанная окружность~$\omega$ касается его стороны~$BC$ в~точке $A_{1}$.
Обозначим через~$\Omega_{A}$ окружность, проходящую через вершины $B$~и~$C$
и~касающуюся~$\omega$, точку касания назовем~$T_{A}$.
Докажите, что точки $T_{A}$, $A_{1}$, $I_{A}$ лежат на~одной прямой.

\item
Остроугольный неравенобедренный треугольник $ABC$ вписан в~окружность~$\Omega$.
Окружность~$\omega$, проходящая через точки $B$ и~$C$, имеет центр~$S$
и~пересекает отрезки $AB$ и~$AC$ второй раз в~точках $P$ и~$Q$ соответственно.
Отрезки $BQ$ и~$CP$ пересекаются в~точке~$R$.
Луч~$SR$ пересекает дугу~$BAC$ окружности~$\Omega$ в~точке~$X$.
Докажите, что центры вписанных окружностей треугольников $XBQ$ и~$XCP$
совпадают.
% https://artofproblemsolving.com/community/c6h420429p2374813

\end{problems}

