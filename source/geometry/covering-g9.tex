% $date: 2018-04-10
% $timetable:
%   g9:
%     2018-04-10:
%       2:

% $build$matter[print]: [[.], [.]]

\worksheet*{Покрытия}

% $authors:
% - Александр Давидович Блинков

\begin{problems}

\item
В~четырехугольнике длины всех сторон и~обеих диагоналей меньше, чем $1$.
Докажите, что его можно поместить в~круг радиуса $0{,}9$.

\item
Круг радиуса~$1$ покрыт семью одинаковыми кругами.
Докажите, что их радиус не~меньше, чем $0{,}5$.

\item
Два треугольника пересекаются (имеют хотя~бы одну общую точку).
Докажите, что внутри описанной окружности одного из~них или на~ее границе лежит
хотя~бы одна вершина другого.

\item
Докажите, что любой треугольник можно разрезать на~три меньших треугольника
так, чтобы каждую из~получившихся частей можно было покрыть двумя другими.

\item
На~столе лежат $5$ одинаковых бумажных треугольников.
Каждый разрешается сдвигать в~любом направлении, но~не~поворачивать.
\\
\subproblem
Верно~ли, что при любом изначальном расположении каждый из~этих треугольников
можно накрыть четырьмя другими?
\\
\subproblem
Ответьте на~тот~же вопрос, если лежащие треугольники~--- равносторонние.

\item
В~четырех заданных точках на~плоскости расположены прожекторы, каждый
из~которых может освещать прямой угол (включая границы).
Стороны этих углов могут быть направлены на~север, юг, запад или восток.
Можно~ли направить эти прожекторы так, чтобы они осветили всю плоскость?

\item
Дан выпуклый пятиугольник, все углы которого тупые.
Докажите, что в~нем найдутся такие две диагонали, что круги, построенные на~них
как на~диаметрах, полностью покроют весь пятиугольник.

\item
Длина проекции фигуры~$F$ на~любую прямую не~превосходит $1$.
Верно~ли, что $F$ можно накрыть кругом диаметра:
\\
\subproblem $1$;
\qquad
\subproblem $1{,}5$?

\item
На~плоскости нарисованы $100$ кругов, любые два из~которых имеют общую точку
(возможно граничную).
Докажите, что найдется точка, принадлежащая не~менее, чем пятнадцати кругам.

\item
На~плоскости расположен круг.
Какое наименьшее количество прямых надо провести, чтобы симметрично отражая
данный круг относительно этих прямых (в~любом порядке и~конечное количество
раз) можно было накрыть им любую точку плоскости?

\end{problems}

