% $date: 2018-04-11
% $timetable:
%   g9:
%     2018-04-11:
%       2:

% $build$matter[print]: [[.], [.]]

\worksheet*{Вокруг площадей}

% $authors:
% - Александр Давидович Блинков

\begin{problems}

\item
В~квадрате $ABCD$ точки $E$ и~$F$~--- середины сторон $BC$ и~$CD$
соответственно.
Отрезки $AE$ и~$BF$ пересекаются в~точке~$G$.
Сравните площади треугольника $AGF$ и~четырехугольника $GECF$.

\item
Дан четырехугольник $ABCD$ площади $1$.
Из~его внутренней точки $O$~опущены перпендикуляры $OK$, $OL$, $OM$ и~$ON$
на~стороны $AB$, $BC$, $CD$ и~$DA$ соответственно.
Известно, что $AK \geq KB$, $BL \geq LC$, $CM \geq MD$ и~$DN \geq NA$.
Найдите площадь четырехугольника $KLMN$.

\item
Диагональ~$BD$ вписанного четырехугольника $ABCD$ является биссектрисой
угла $ABC$.
Найдите площадь $ABCD$, если $BD = 6\,\text{см}$, $\angle ABC = 60^{\circ}$.

\item
Диаметр~$PQ$ и~перпендикулярная ему хорда~$MN$ пересекаются в~точке~$A$.
Точка~$C$ лежит на~окружности, а~точка~$B$~--- внутри окружности, причем
$BC \parallel PQ$ и~$BC = MA$.
Из~точек $A$ и~$B$ опущены перпендикуляры $AK$ и~$BL$ на~прямую~$CQ$.
Докажите, что треугольники $ACK$ и~$BCL$ равновелики.

\item
Биссектриса~$AL$ треугольника $ABC$ пересекает описанную окружность
в~точке~$W$.
Точки $M$ и~$N$~--- проекции точки~$L$ на~$AB$ и~$AC$.
Докажите, что четырехугольник $AMWN$ и~треугольник $ABC$ равновелики.

\item
\begin{minipage}[t][][t]{0.77\linewidth}
Через точку~$P$ проведены три прямые, параллельные сторонам треугольника $ABC$
(см. рисунок).
Докажите, что треугольники $A_{1}B_{1}C_{1}$ и~$A_{2}B_{2}C_{2}$ равновелики.
\end{minipage}\hfill
\begin{minipage}[t][][b]{0.20\linewidth}
    \vspace{-3ex}%
    \hspace{-0.3\linewidth}\hfill\jeolmfigure{triangle}
\end{minipage}

\item
$ABCD$~--- выпуклый четырехугольник площади~$S$.
Угол между прямыми $AB$ и~$CD$ равен $\alpha$, а~угол между прямыми $AD$ и~$BC$
равен $\beta$.
Докажите неравенство:
\[
    \frac{AB \cdot CD \cdot \sin(\alpha) + AD \cdot BC \cdot \sin(\beta)}{2}
\leq
    S
\leq
    \frac{AB \cdot CD + AD \cdot BC}{2}
\; . \]

\item
В~треугольнике $ABC$ угол~$B$ равен $60^{\circ}$.
Точка~$D$ внутри треугольника такова, что
$\angle ADB = \angle ADC = \angle BDC$.
\\
\subproblem
Найдите наименьшее значение площади треугольника $ABC$, если $BD = a$;
\\
\subproblem
В~каком случае оно достигается?

\item
В~треугольнике $ABC$ точка~$D$~--- середина стороны~$AB$.
Можно~ли так расположить точки $E$ и~$F$ на~сторонах $AC$ и~$BC$
соответственно, чтобы площадь треугольника $DEF$ оказалась больше суммы
площадей треугольников $AED$ и~$BFD$?

\item
На~плоскости заданы $n$-угольник $A_{1}A_{2}{\ldots}A_{n}$ площади~$S$
и~произвольная точка~$P$.
Повернув точку~$P$ на~один и~тот~же заданный угол $\alpha$ относительно каждой
из~вершин данного многоугольника, получим новый $n$-угольник.
Найдите его площадь.

\end{problems}

