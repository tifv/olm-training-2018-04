% $date: 2018-04-10
% $timetable:
%   g9:
%     2018-04-10:
%       1:

% $caption:
%   Ортоцентр, середина стороны, точка пересечания касательных и… ещё одна
%   точка!

% $build$matter[print]: [[.], [.]]

\worksheet*{%
Ортоцентр, середина стороны, точка пересечения касательных
и\ldots\ еще одна точка!}

% $authors:
% - Александр Давидович Блинков

Пусть $AA_{1}$ и~$BB_{1}$~--- высоты остроугольного неравнобедренного
треугольника $ABC$, $H$~--- его ортоцентр, $M$~--- середина~$AB$.
Окружности $\omega$ с~центром~$O$ и~$\omega_{1}$ с~центром~$O_{1}$, описанные
около треугольников $ABC$ и~$A_{1}B_{1} C$ соответственно, вторично
пересекаются в~точке~$P$.

\begin{problems}

\item
Докажите, что точки $M$, $H$ и~$P$ лежат на~одной прямой.

\item
Докажите, что:
\\
\subproblem
окружности, описанные около треугольников $AMA_{1}$ и~$BMB_{1}$, проходят через
точку~$P$;
\\
\subproblem
$PM$~--- биссектриса углов $APA_{1}$ и~$BPB_{1}$;
\\
\subproblem
прямая~$PA$ проходит через точку, симметричную точке~$A_{1}$ относительно
прямой~$CH$.

\item
Пусть $L_{1}$ и~$L_{2}$~--- вторые точки пересечения окружности, описанной
около треугольника $AMA_{1}$, с~прямыми~$BC$ и~$AC$ соответственно,
а~$K_{1}$ и~$K_{2}$~--- вторые точки пересечения окружности, описанной около
треугольника $BMB_{1}$, с~прямыми $AC$ и~$BC$ соответственно.
Докажите, что:
\\
\subproblem $L_{1}$, $K_{1}$, $M$ и~$O$ лежат на~одной прямой;
\\
\subproblem $L_{2}$, $K_{2}$, $M$ и~$O_{1}$ лежат на~одной прямой;
\\
\subproblem $L_{1}$, $K_{1}$, $L_{2}$ и~$K_{2}$ лежат на~одной окружности;
\\
\subproblem прямые $L_{1} L_{2}$, $K_{1} K_{2}$ и~$PM$ пересекаются в~одной точке.

\item
Пусть прямые $A_{1}B_{1}$ и~$AB$ пересекаются в~точке~$S$, $R$~--- середина
отрезка~$CM$.
Докажите, что:
\\
\subproblem точки $C$, $P$ и~$S$ лежат на~одной прямой;
\\
\subproblem прямые $SH$ и~$CM$ перпендикулярны;
\\
\subproblem прямые $OR$ и~$SC$ перпендикулярны.

\item
Пусть касательные к~окружности~$\omega$, проведенные в~точках $A$ и~$B$,
пересекают прямую $A_{1}B_{1}$ в~точках $X$ и~$Y$ соответственно и~пересекаются
в~точке~$Z$.
Докажите, что:
\\
\subproblem
точка~$M$~--- центр вписанной окружности треугольника $XYZ$;
\\
\subproblem
окружности, описанные около треугольников $AMA_{1}$ и~$BMB_{1}$, проходят через
точки $X$ и~$Y$ соответственно;
\\
\subproblem
прямые $MH$, $A_{1}B_{1}$ и~$ZC_{1}$ пересекаются в~одной точке
($C_{1}$~--- точка пересечения $CH$ и~$AB$).
\\
\subproblem
прямая~$ZP$ проходит через точку $H_{c}$, симметричную $H$ относительно
стороны~$AB$.
\\
\subproblem
описанные окружности треугольников $ABC$ и~$XYZ$ касаются в~точке~$P$.
\\
\subproblem
прямые $AP$, $BC$ и~$ZC_{1}$ пересекаются в~одной точке.

\end{problems}

