% $date: 2018-04-06
% $timetable:
%   g8:
%     2018-04-06:
%       1:

\worksheet*{Для тех кому скучно: препарирование четвертой задачи}

% $authors:
% - Фёдор Львович Бахарев

\begin{problems}

\item
Пусть $O$~--- центр описанной окружности остроугольного треугольника $ABC$,
$T$~--- центр описанной окружности треугольника $AOC$, $M$~--- середина~$AC$.
На~сторонах $AB$ и~$BC$ выбраны точки $D$ и~$E$ соответственно так, что
$\angle BDM = \angle BEM = \angle B$.
Докажите, что $BT \perp DE$.
% Финал 03-04, 8

\item
В~треугольнике $ABC$ проведена биссектриса~$BB_{1}$.
Перпендикуляр, опущенный из~точки~$B_{1}$ на~$BC$, пересекает дугу~$BC$
описанной окружности треугольника $ABC$ в~точке~$K$
Перпендикуляр опущенный из~точки~$B$ на~$AK$ пересекает $AC$ в~точке~$L$.
Докажите что точки $K$, $L$ и~середина дуги~$AC$ (не~содержащей точку~$B$)
лежат на~одной прямой.
% Финал 06-07, 4

\item
Дан треугольник $ABC$.
Окружность~$\omega$ касается описанной окружности~$\Omega$ треугольника $ABC$
в~точке~$A$, пересекает сторону~$AB$ в~точке~$K$, а~также пересекает
сторону~$BC$.
Касательная~$CL$ к~окружности~$\omega$ такова, что отрезок~$KL$ пересекает
сторону~$BC$ в~точке~$T$.
Докажите, что отрезок~$BT$ равен по~длине касательной, проведенной из~точки~$B$
к~$\omega$.
% Финал 05-06, 4

\item
Точка~$M$~--- середина стороны~$AC$ остроугольного треугольника $ABC$,
в~котором $AB > BC$.
Окружность~$\Omega$ описана около треугольника $ABC$.
Касательные к~$\Omega$, проведенные в~точках $A$ и~$C$, пересекаются
в~точке~$P$.
Отрезки $BP$ и~$AC$ пересекаются в~точке~$S$.
Пусть $AD$~--- высота треугольника $BPA$.
Окружность~$\omega$, описанная около треугольника $CSD$, второй раз пересекает
окружность~$\Omega$ в~точке~$K$.
Докажите, что $\angle CKM = 90^\circ$.
% Финал 13-14, 4

\item
На~сторонах $AP$ и~$PD$ остроугольного треугольника $APD$ выбраны
соответственно точки $B$ и~$C$.
Диагонали четырехугольника $ABCD$ пересекаются в~точке~$Q$.
Точки $H_{1}$ и~$H_{2}$ являются ортоцентрами треугольников $APD$ и~$BPC$
соответственно.
Докажите, что если прямая~$H_{1}H_{2}$ проходит через точку~$X$ пересечения
описанных окружностей треугольников $ABQ$ и~$CDQ$, то~она проходит и~через
точку~$Y$ пересечения описанных окружностей треугольников $BQC$ и~$AQD$.
($X \neq Q$, $Y \neq Q$.)
% Финал 02-03, 8

\end{problems}

