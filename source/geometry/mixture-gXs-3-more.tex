% $date: 2018-04-06
% $timetable:
%   gXs:
%     2018-04-06:
%       3:

\worksheet*{Еще пять задач}

% $authors:
% - Фёдор Львович Бахарев

\begin{problems}

\item
Даны шесть точек $A$, $B$, $C$, $D$, $E$ и~$F$ такие, что
треугольники $BCD$, $ECA$ и~$BFA$ подоны и~одинаково ориентированы.
Точка~$I$~--- инцентр треугольника $ABC$.
Докажите, что центры описанных окружностей треугольников $AID$, $BIE$ и~$CIF$
коллинеарны.
% https://artofproblemsolving.com/community/c6h1618841p10139513

\item
Треугольник $ABC$ вписан в~окружность с~центром~$O$, а~его
высоты $AD$, $BE$ и~$CF$ пересекаются в~точке~$H$.
Для произвольной точки~$P$ на~прямой~$OH$ построим вторые точки пересечения
прямых $DP$, $EP$, $FP$ с~окружностью девяти точек:
получим точки $A_{1}$, $B_{1}$ и~$C_{1}$, соответственно.
Пусть $A_{2}$, $B_{2}$ и~$C_{2}$~--- точки, симметричные точкам $A$, $B$ и~$C$
относительно $A_{1}$, $B_{1}$ и~$C_{1}$ соответственно.
Докажите, что точки $H$, $A_{2}$, $B_{2}$, $C_{2}$ лежат на~одной окружности
с~центром на~$OH$.
% https://artofproblemsolving.com/community/c6h1618897p10129391

\item
Дан треугольник $ABC$ и~его ортоцентр~$H$.
Пусть $L_{a}$, $L_{b}$, $L_{c}$ и~$L$~--- точки Лемуана
треугольников $HBC$, $HCA$, $HAB$ и~$ABC$ соответственно.
Докажите, что прямые $AL_{a}$, $BL_{b}$, $CL_{c}$ и~$HL$ пересекаются
в~одной точке.
% https://artofproblemsolving.com/community/c374081h1579486p9744743

\item
Медианы из~вершин $B$ и~$C$ треугольника $ABC$ пересекают его описанную
окружность~$\omega$ в~точках $D$ и~$E$ соответственно.
Пусть $O_{1}$~--- центр окружности, проходящей через точку~$D$ и~касающейся
прямой~$AC$ в~точке~$C$, а~$O_{2}$~--- центр окружности, проходящей через
точку~$E$ и~касающейся прямой~$AB$ в~точке~$B$.
Докажите, что прямая~$O_{1}O_{2}$ проходит через центр окружности девяти точек
треугольника $ABC$.
% https://artofproblemsolving.com/community/c6h545088

\item
Окружность~$\omega$ касается сторон $AB$ и~$AC$ треугольника $ABC$
в~точках $D$ и~$E$ соотвественно, причем $BD + CE < BC$.
Точки $F$ и~$G$ лежат на~$BC$ и~таковы, что $BF = BD$, $CG = CE$.
Пусть $DG$ и~$EF$ пересекаются в~точке~$K$, а~точка~$L$ на~малой дуге~$DE$
окружности~$\omega$ такова, что касательная в~точке~$L$ к~окружности~$\omega$
параллельна $BC$.
Докажите, что инцентр треугольника $ABC$ лежит на~прямой~$KL$.
% https://artofproblemsolving.com/community/c6h1568534p9672203

\end{problems}

