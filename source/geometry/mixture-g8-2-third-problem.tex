% $date: 2018-04-04
% $timetable:
%   g8:
%     2018-04-04:
%       2:

\worksheet*{Трагедия третьей задачи}

% $authors:
% - Фёдор Львович Бахарев

\begin{problems}
    \ifdefined\vartriangle
        \def\triangle{\mathord{\vartriangle}}
    \fi

\item
На~сторонах остроугольного треугольника $ABC$ вне него построены квадраты
$CAKL$ и $CBMN$.
Прямая~$CN$ пересекает отрезок~$AK$ в~точке $X$, а~прямая~$CL$ пересекает
отрезок~$BM$ в~точке~$Y$.
Точка~$P$, лежащая внутри треугольника $ABC$, является точкой пересечения
описанных окружностей треугольников $KXN$ и~$LYM$.
Найдите еще пару <<хороших>> точек, кроме точки~$P$, лежащих на~радикальной оси
указанных окружностей.
Точка~$S$~--- середина отрезка~$AB$.
Докажите, что $\angle ACS = \angle BCP$.
% Финал 12-13, 7

\item
Дан параллелограмм $ABCD$ с~тупым углом~$A$.
Точка~$H$~--- основание перпендикуляра, опущенного из~точки $A$ на~$BC$.
Продолжение медианы~$CM$ треугольника $ABC$ пересекает описанную около него
окружность в~точке~$K$.
Найдите <<хорошую>> точку на~окружности, описанной около треугольника $HCD$.
Докажите, что точки $K$, $H$, $C$ и~$D$ лежат на~одной окружности.
% Финал 11-12, 3

\item
\subproblem
Медианы треугольника $\triangle$ параллельны сторонам
треугольника $\triangle'$.
Докажите, что медианы треугольника $\triangle'$ параллельны сторонам
треугольника $\triangle$.
\\
\subproblem
В~неравнобедренном треугольнике $ABC$ точки $H$ и~$M$~--- точки пересечения
высот и~медиан соответственно.
Через вершины $A$, $B$ и~$C$ проведены прямые, перпендикулярные
прямым $AM$, $BM$, $CM$ соответственно.
Докажите, что точка пересечения медиан треугольника, образованного проведёнными
прямыми, лежит на~прямой~$MH$.
% Финал 07-08, 3

\item
Пусть $ABC$~--- правильный треугольник.
На~его стороне~$AC$ выбрана точка~$T$, а~на~дугах $AB$ и~$BC$ его описанной
окружности выбраны точки $M$ и~$N$ соответственно так, что $MT \parallel BC$
и~$NT \parallel AB$.
Отрезки $AN$ и~$MT$ пересекаются в~точке~$X$, а~отрезки $CM$ и~$NT$~---
в~точке~$Y$.
Найдите на~картинке вписанные четырехугольники и~докажите, что периметры
многоугольников $AXYC$ и~$XMBNY$ равны.
% Финал 10-11, 7

\item
Остроугольный треугольник $ABC$ ($AB < AC$) вписан в~окружность~$\Omega$.
Пусть $M$~--- точка пересечения его медиан, а~$AH$~--- высота.
Луч~$MH$ пересекает $\Omega$ в~точке~$A'$.
Докажите, что описанная окружность треугольника $A'HB$ касается прямой~$AB$.
% Финал 14-15, 7
% углы, окружность девяти точек, гомотетия, угол между касательной и хордой

\item
Внутри параллелограмма $ABCD$ выбрана точка~$K$ таким образом, что середина
отрезка~$AD$ равноудалена от~точек $K$ и~$C$, а~середина отрезка~$CD$
равноудалена от~точек $K$ и~$A$.
Точка~$N$~--- середина отрезка~$BK$.
Докажите, что углы $NAK$ и~$NCK$ равны.
% Финал 00-01, 3

\item
Пусть $O$~--- центр описанной окружности~$\omega$ остроугольного
треугольника $ABC$.
Окружность~$\omega_{1}$ с~центром~$K$ проходит через точки $A$, $O$ и~$C$
и~пересекает стороны $AB$ и~$BC$ в~точках $M$ и~$N$.
Известно, что точки $L$ и~$K$ симметричны относительно прямой~$MN$.
Докажите, что $BL \perp AC$.
% Финал 99-00, 3

\item
На~медиане~$CD$ треугольника $ABC$ отмечена точка~$E$.
Окружность~$S_{1}$, проходящая через точку~$E$ и~касающаяся прямой~$AB$
в~точке~$A$, пересекает сторону~$AC$ в~точке~$M$.
Окружность~$S_{2}$, проходящая через точку~$E$ и~касающаяся прямой~$AB$
в~точке~$B$, пересекает сторону~$BC$ в~точке~$N$.
Докажите, что описанная окружность треугольника $CMN$ касается
окружностей $S_{1}$ и~$S_{2}$.
% Финал 99-00, 7

\end{problems}

