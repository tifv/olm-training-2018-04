% $date: 2018-04-12
% $timetable:
%   gXs:
%     2018-04-12:
%       1:

\worksheet*{Стереометрия}

% $authors:
% - Андрей Юрьевич Кушнир

\begin{problems}

\item
Из~произвольной точки внутри данного куба опущены перпендикуляры на~плоскости
его граней.
Полученные шесть отрезков являются диагоналями других кубов.
Рассмотрим шесть сфер, каждая из~которых касается всех рёбер соответствующего
куба.
Докажите, что все эти сферы имеют общую касательную прямую.

\item
В~грани $BCD$, $ACD$, $ABD$, $ABC$ тетраэдра $ABCD$ вписаны окружности
$\omega_{A}$, $\omega_{B}$, $\omega_{C}$, $\omega_{D}$ соответственно.
Через середину отрезка, соединяющего точки касания $\omega_{A}$, $\omega_{B}$
с~ребром~$CD$, провели плоскость~$\Pi_{AB}$, перпендикулярную линии
центров $\omega_{A}$, $\omega_{B}$.
Аналогично определены
плоскости $\Pi_{AC}$, $\Pi_{AD}$, $\Pi_{BC}$, $\Pi_{BD}$, $\Pi_{CD}$.
Докажите, что все шесть построенных плоскостей имеют общую точку.

\item
Касательную плоскость в~точке~$A$ к~описанной сфере тетраэдра $ABCD$ пересекли
с~плоскостями граней $ABC$, $ACD$, $ADB$.
Докажите, что три полученные прямые при пересечении образуют шесть равных
углов, если и~только если выполнено соотношение:
\(
    AB \cdot CD = AC \cdot BD = AD \cdot BC
\).

\item
Пусть $A$ и~$B$~--- различные точки, принадлежащие линии пересечения
перпендикулярных плоскостей $\pi_{1}$ и~$\pi_{2}$.
Точка~$C$ принадлежит плоскости $\pi_{2}$, но~не~принадлежит $\pi_{1}$.
Обозначим через $P$ точку пересечения биссектрисы угла $ACB$ с~прямой~$AB$
и~через $\omega$ окружность с~диаметром~$AB$ в~плоскости~$\pi_{1}$.
Плоскость~$\pi_{3}$, содержащая $CP$, пересекает окружность~$\omega$
в~точках $D$~и~$E$.
Докажите, что $CP$~--- биссектриса угла $DCE$.

\item
В~треугольной пирамиде $ABCD$ все плоские углы при вершинах~--- не~прямые,
а~точки пересечения высот в~треугольниках $ABC$, $ABD$, $ACD$ лежат на~одной
прямой.
Докажите, что центр описанной сферы пирамиды лежит в~плоскости, проходящей
через середины ребер $AB$, $AC$, $AD$.

\item
Вписанная в~тетраэдр $ABCD$ сфера касается его
граней $ABC$, $ABD$, $ACD$ и~$BCD$ в~точках $D_{1}$, $C_{1}$, $B_{1}$ и~$A_{1}$
соответственно.
Рассмотрим плоскость, равноудаленную от~точки~$A$ и~плоскости $B_{1}C_{1}D_{1}$
и~три другие аналогично построенные плоскости.
Докажите, что тетраэдр, образованный этими четырьмя плоскостями, имеет тот~же
центр описанной сферы, что и~тетраэдр $ABCD$.

\item
Тетраэдр $ABCD$ с~остроугольными гранями вписан в~сферу с~центром~$O$.
Прямая, проходящая через точку~$O$ перпендикулярно плоскости $ABC$, пересекает
сферу в~точке~$E$ такой, что $D$ и~$E$ лежат по~разные стороны относительно
плоскости $ABC$.
Прямая~$DE$ пересекает плоскость $ABC$ в~точке~$F$, лежащей внутри
треугольника $ABC$.
Оказалось, что $\angle ADE = \angle BDE$, $AF \neq BF$.
Докажите, что $\angle ACB = \frac{1}{2} \angle AFB$.

\end{problems}

