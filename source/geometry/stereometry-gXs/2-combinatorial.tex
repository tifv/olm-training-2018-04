% $date: 2018-04-13
% $timetable:
%   gXs:
%     2018-04-13:
%       3:

\worksheet*{Комбинаторная стереометрия и~неравенства}

% $authors:
% - Андрей Юрьевич Кушнир

\begin{problems}
    \ifdefined\mathup
        \def\piconst{\mathup{\muppi}}%
    \else
        \def\piconst{\uppi}%
    \fi

\item
Высота и~радиус основания цилиндра равны $1$.
Каким наименьшим числом шаров радиуса~$1$ можно целиком покрыть этот цилиндр?

\item
Точка~$O$ лежит в~основании $A_{1}A_{2} \ldots A_{n}$ пирамиды
$SA_{1}A_{2}{\ldots}A_{n}$, причем
\(
    SA_{1} = SA_{2} = \ldots = SA_{n}
\) и~\(
    \angle SA_{1}O = \angle SA_{2}O = \ldots = \angle SA_{n}O
\).
При каком наименьшем значении $n$ отсюда следует, что $SO$~--- высота пирамиды?

%\item
%Верно~ли, что в~произвольном тетраэдре возможно выбрать пару скрещивающихся
%ребер пирамиды так, чтобы любая точка тетраэдра лежала хотя~бы в~одном
%из~шаров, построенных на~выбранных ребрах как на~диаметрах?

\item
Могут~ли $4$~центра вписанных в~грани тетраэдра окружностей лежать в~одной
плоскости?

\item
В~тетраэдр $ABCD$, длины всех ребер которого не~более $100$, можно поместить
две непересекающиеся сферы диаметра~$1$.
Докажите, что в~него можно поместить одну сферу диаметра $1{,}01$.

\item
Дана $n$-угольная призма $A_{1}A_{2}{\ldots}A_{n} B_{1}B_{2}{\ldots}B_{n}$.
Внутри призмы взята произвольная точка~$O$.
На~лучах $OA_{1}$, $OA_{2}$, \ldots, $OA_{n}$,
$OB_{1}$, $OB_{2}$, \ldots, $OB_{n}$ отмечены
точки $X_{1}$, $X_{2}$, \ldots, $X_{n}$, $Y_{1}$, $Y_{2}$, \ldots, $Y_{n}$
соответственно.
Докажите, не~существует выпуклого многогранника, в~множестве ребер которого были~бы отрезки
$X_{1}Y_{2}$, $X_{2}Y_{3}$, $X_{3}Y_{4}$, \ldots, $X_{n-1}Y_{n}$, $X_{n}Y_{1}$.

\item
Сумма мер телесных углов при вершинах выпуклого многогранника равна $\piconst$.
Докажите, что существует замкнутый маршрут по~его ребрам, проходящий через
каждую вершину ровно по~одному разу.
Мера телесного угла определена как площадь части единичной сферы с~центром
в~вершине угла, съеденной внутренностью угла.
Площадь поверхности единичной сферы равна $4 \piconst$.
% http://kvant.mccme.ru/pdf/2010/2010-03.pdf

\end{problems}

