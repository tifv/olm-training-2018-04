% $date: 2018-04-04
% $timetable:
%   gXj:
%     2018-04-04:
%       1:

\worksheet*{Ортополюсы и~разнобой}

% $authors:
% - Фёдор Львович Бахарев

\begin{problems}

\item
Из~вершин треугольника $ABC$ опустим перпендикуляры $AA'$, $BB'$ и~$CC'$
на~прямую~$\ell$.
Докажите, что перпендикуляры из~$A'$ на~$BC$, из~$B'$ на~$CA$ и~из~$C'$ на~$AB$
пересекаются в~одной точке.
Эта точка называется \emph{ортополюсом} прямой~$\ell$ относительно
треугольника $ABC$.
% http://www.artofproblemsolving.com/community/c6h612457
% http://mathworld.wolfram.com/Orthopole.html

\item
Докажите, что ортополюс прямой~$\ell$ относительно треугольника $ABC$ лежит
на~некоторой прямой Симсона, перпендикулярной $\ell$.
Что происходит с~ортополюсом, если прямая~$\ell$ смещается параллельно самой
себе?

\item
Докажите, что если прямая~$\ell$ пересекает описанную окружность
треугольника $ABC$ в~точках $P_{1}$ и~$P_{2}$, то~прямые Симсона
точек $P_{1}$ и~$P_{2}$ пересекаются в~ортополюсе прямой~$\ell$.

\item
Докажите, что если $\ell$ проходит через центр описанной окружности
треугольника $ABC$, то~ее ортополюс лежит на~окружности девяти точек
треугольника.

\item
Четырехугольник $ABCD$ вписан.
Докажите, что ортополюсы прямой~$\ell$ относительно
треугольников $ABC$, $BCD$, $CDA$ и~$DAB$ лежат на~одной прямой.

\item
Пусть $O_{a}$, $O_{b}$, $O_{c}$, $O_{d}$ центры описанных окружностей
треугольников $BCD$, $CDA$, $DAB$ и~$ABC$ соответственно.
Пусть $T_{a}$, $T_{b}$, $T_{c}$, $T_{d}$ ортополюсы
прямых $O_{a}P$, $O_{b}P$, $O_{c}P$ и~$O_{d}P$ относительно
треугольников $BCD$, $CDA$, $DAB$, $ABC$ соответственно.
Тогда точки $T_{a}$, $T_{b}$, $T_{c}$ и~$T_{d}$ лежат на~одной окружности.

\item
Вписаная окружность треугольника $ABC$ касается стороны~$BC$ в~точке~$A_{1}$.
Прямая~$AA_{1}$ вторично пересекает окружность в~точке~$P$.
Пусть прямые $CP$ и~$BP$ пересекают окружность повторно в~точках $N$ и~$M$
соответственно.
Докажите, что $AA_{1}$, $BN$ и~$CM$ пересекаются в~одной точке.
% https://www.artofproblemsolving.com/community/c6h2619p7954

\item
В~треугольнике $ABC$ даны центр вписанной окружности $I$ и~центр описанной
окружности~$O$.
Прямая~$\ell$ параллельна $BC$ и~касается вписаной окружности и~пересекает
прямую~$IO$ в~точке~$X$.
Точка~$Y$ на~$\ell$ такова, что $\angle YIO = 90^{\circ}$.
Докажите, что точки $A$, $X$, $O$, $Y$ лежат на~одной окружности.
% http://www.artofproblemsolving.com/community/q2h598536p3551853

\item
Пусть $AH$~--- высота треугольника $ABC$ и~точки~$H'$ симметрична $H$
относительно середины стороны~$BC$.
Касательные в~точках $B$ и~$C$ к~окружности, описанной около треугольника,
пересекаются в~точке~$X$.
Перпендикуляр к~прямой~$XH'$ в~точке~$H'$ пересекает прямые $AB$ и~$AC$
в~точках $Y$ и~$Z$ соответственно.
Докажите, что $\angle ZXC = \angle YXB$.
% http://www.artofproblemsolving.com/community/c6h1095220p4902494

\end{problems}

