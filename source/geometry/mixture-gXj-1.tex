% $date: 2018-04-03
% $timetable:
%   gXj:
%     2018-04-03:
%       2:

\worksheet*{Разнобой}

% $authors:
% - Фёдор Львович Бахарев

\begin{problems}

\item
Точка~$X$ вне треугольника $ABC$ такова, что $A$ лежит внутри
треугольника $BXC$.
При этом $2 \angle XBA = \angle ACB$, $2 \angle XCA = \angle ABC$.
Докажите, что центры описанной и~вневписанной со~стороны~$BC$ окружностей
треугольника $ABC$ и~точка~$X$ лежат на~одной прямой.

\item
В~четырехугольнике $ABCD$ выполнено соотношение $\angle ACB = \angle ACD$.
Точка~$T$ внутри четырехугольника $ABCD$ такова, что
$\angle ADC - \angle ATB = \angle BAC$
и~$\angle ABC - \angle ATD = \angle CAD$.
Докажите, что $\angle BAT = \angle DAC$.

\item
Отрезки $BX$ и~$CY$ касаются описанной окружности треугольника $ABC$, причем
$AB = BX$, $AC = CY$ и~точки $X$, $Y$ и~$A$ лежат по~одну сторону
от~прямой~$BC$.
Докажите, что если $I$~--- центр вписанной окружности треугольника $ABC$,
то~$\angle BAC + \angle XIY = 180^{\circ}$.

\item
В~трапеции $ABCD$ выполнены соотношения $AB < CD$, $AB \parallel CD$
и~$P = AD \cap BC$.
Предположим, что точка~$Q$ внутри $ABCD$ такова, что
$\angle QAB = \angle QDC = 90^{\circ} - \angle BQC$.
Докажите, что $\angle PQA = 2 \angle QCD$.

\item
В~остроугольном треугольнике $ABC$ высоты $BF$ и~$CE$ пересекаются
в~ортоцентре~$H$, а~точка~$M$~--- середина стороны~$BC$.
Пусть точка~$X$ на~прямой~$EF$ такова, что $\angle XMH = \angle HAM$, причем
точки $A$ и~$X$ лежат по~разные стороны от~$MH$.
Докажите, что $AH$ делит $MX$ пополам.

\item
Пусть $ABC$ такой треугольник, что $\angle C = 2 \angle B$, а~$\omega$~--- его
описанная окружность.
Касательная в~точке~$A$ к~$\omega$ пересекает $BC$ в~точке~$E$.
Окружность~$\Omega$ проходит через вершину~$B$ и~касается стороны~$AC$
в~точке~$C$.
Пусть $\Omega \cap AB = F$, а~точка $K \in \Omega$ такова, что $EK$ касается $\Omega$,
и~точки $A$ и~$K$ лежат по~разные стороны от~$BC$.
Докажите, что если $M$~--- середина дуги~$BC$, не~содержащей точку~$A$,
окружности~$\omega$, то~четырехугольника $AFMK$ вписаный.

\item
Диагонали вписанного четырехугольника $ABCD$ пересекаются в~точке~$P$.
Окружность~$\Gamma$ касается продолжений $AB$, $BC$, $AD$, $DC$
в~точках $X$, $Y$, $Z$, $T$ соответственно.
Окружность~$\Omega$ проходит через $A$ и~$B$, и~касается внешне
окружности~$\Gamma$ в~точке~$S$.
Докажите, что $SP \perp ST$.

\item
На~плоскости даны $n$ различных точек и~круг радиуса~$r$ с~центром в~точке~$O$.
Хотя~бы одна точка лежит внутри круга.
На~каждом шаге мы передвигаем $O$ в~центр масс точек, попавших внутрь круга.
Докажите, что процесс стабилизируется.

\end{problems}

