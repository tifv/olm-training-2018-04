% $date: 2018-04-03
% $timetable:
%   g8:
%     2018-04-03:
%       1:

\worksheet*{Как решить вторую задачу?}

% $authors:
% - Фёдор Львович Бахарев

\begin{problems}

\item
На~сторонах $AB$, $BC$, $CA$ треугольника $ABC$ выбраны точки $P$, $Q$, $R$
соответственно таким образом, что $AP = CQ$ и~четырёхугольник $RPBQ$~---
вписанный.
Касательные к~описанной окружности треугольника $ABC$ в~точках $A$ и~$C$
пересекают прямые $RP$ и~$RQ$ в~точках $X$ и~$Y$ соответственно.
Докажите, что $RX = RY$.
% Финал 05-06, 6
% счет углов и равенство треугольников! угол между касательной и хордой

\item
На~одной стороне угла с~вершиной~$O$ взята точка~$A$, а~на~другой~---
точки $B$ и~$C$, причём точка~$B$ лежит между $O$ и~$C$.
Проведена окружность с~центром~$O_{1}$, вписанная в~треугольник $OAB$,
и~окружность с~центром~$O_{2}$, касающаяся стороны $AC$ и~продолжений сторон
$OA$ и~$OC$ треугольника $AOC$.
Докажите, что если $O_{1}A = O_{2}A$, то~треугольник $ABC$~--- равнобедренный.
% Финал 01-02, 2
% счет углов, знание стандартных соотношений в треугольниках

\item
Вписанная окружность касается сторон $AB$ и~$AC$ треугольника $ABC$
в~точках $X$ и~$Y$ соответственно.
Точка~$K$~--- середина дуги~$AB$ описанной окружности треугольника $ABC$
(не~содержащей точки~$C$).
Оказалось, что прямая~$XY$ делит отрезок~$AK$ пополам.
Чему может быть равен угол $BAC$?
% Финал 07-08, 6
% трезубец

\item
Серединный перпендикуляр к~стороне~$AC$ неравнобедренного остроугольного
треугольника $ABC$ пересекает прямые $AB$ и~$BC$ в~точках $B_{1}$ и~$B_{2}$
соответственно, а~серединный перпендикуляр к~стороне~$AB$ пересекает
прямые $AC$ и~$BC$ в~точках $C_{1}$ и~$C_{2}$ соответственно.
Окружности, описанные около треугольников $BB_{1}B_{2}$ и~$CC_{1}C_{2}$
пересекаются в~точках $P$ и~$Q$.
Докажите, что центр окружности, описанной около треугольника $ABC$, лежит
на~прямой $PQ$.
% Регион 12-13, 7
% счет углов плюс степень точки

\item
Трапеция $ABCD$ с~основаниями $AB$ и~$CD$ вписана в~окружность~$\Omega$.
Окружность~$\omega$ проходит через точки $C$, $D$ и~пересекает отрезки $CA$,
$CB$ в~точках $A_{1}$, $B_{1}$ соответственно.
Точки $A_{2}$ и~$B_{2}$ симметричны точкам $A_{1}$ и~$B_{1}$ относительно
середин отрезков $CA$ и~$CB$ соответственно.
Докажите, что точки $A$, $B$, $A_{2}$ и~$B_{2}$ лежат на~одной окружности.
% Финал 13-14, 6
% степень точки, в обратную сторону

\item
Четырехугольник $ABCD$ описан около окружности.
Биссектрисы внешних углов $A$ и~$B$ пересекаются в~точке~$K$,
внешних углов $B$ и~$C$~--- в~точке~$L$,
внешних углов $C$ и~$D$~--- в~точке $M$,
внешних углов $D$ и~$A$~--- в~точке $N$.
Пусть $K_{1}$, $L_{1}$, $M_{1}$, $N_{1}$~--- точки пересечения высот
треугольников $ABK$, $BCL$, $CDM$, $DAN$ соответственно.
Докажите, что четырехугольник $K_{1}L_{1}M_{1}N_{1}$~--- параллелограмм.
% Финал 03-04, 2
% анализ части картинки, параллельность

\item
Вписанная окружность треугольника $ABC$ касается сторон $AB$ и~$BC$ в~точках
$P$ и~$Q$.
Прямая~$PQ$ пересекает описанную окружность треугольника $ABC$ в~точках $X$
и~$Y$.
Найдите $\angle ABC$, если $\angle XBY = 135^{\circ}$.
% Туй 04.Мл7 апгрейд.
% жесткость

\end{problems}

