% $date: 2018-04-02
% $timetable:
%   gXs:
%     2018-04-02:
%       2:

\worksheet*{Новые и~старые идеи}

% $authors:
% - Фёдор Львович Бахарев

\begin{problems}

\item
Точка~$X$ вне треугольника $ABC$ такова, что $A$ лежит внутри
треугольника $BXC$.
При этом $2 \angle XBA = \angle ACB$, $2 \angle XCA = \angle ABC$.
Докажите, что центры описанной и~вневписанной со~стороны~$BC$ окружностей
треугольника $ABC$ и~точка~$X$ лежат на~одной прямой.

\item
В~четырехугольнике $ABCD$ выполнено соотношение $\angle ACB = \angle ACD$.
Точка~$T$ внутри четырехугольника $ABCD$ такова, что
$\angle ADC - \angle ATB = \angle BAC$
и~$\angle ABC - \angle ATD = \angle CAD$.
Докажите, что $\angle BAT = \angle DAC$.

\item
Пусть $H$~--- ортоцентр треугольника $ABC$, $P$~--- произвольная точка
на~окружности, описанной около треугольника $ABC$.
Симедиана из~вершины~$A$ треугольника $APH$ пересекает $BC$ в~точке~$X$,
симедиана из~вершины $B$ треугольника $BPH$ пересекает $CA$ в~точке $Y$,
симедиана из~вершины $C$ треугольника $CPH$ пересекает $AB$ в~точке $Z$.
Докажите, что точки $X$, $Y$ и~$Z$ коллинеарны.

\item
Через ортоцентр~$H$ треугольника $ABC$ проведена пара перпендикулярных
прямых $\ell_{1}$ и~$\ell_{2}$.
Пусть $\ell_{1} \cap BC = D$ и~$\ell_{1} \cap AB = Z$,
$\ell_{2} \cap BC = E$ и~$\ell_{2} \cap AC = X$.
Прямая, проходящая через $D$ параллельно $AC$, пересекает прямую, проходящую
через $E$ параллельно $AB$, в~точке~$Y$.
Докажите, что точки $X$, $Y$ и~$Z$ коллинеарны.

\item
Точки $A$, $B$, $C$ и~$D$ лежат на~прямой~$\ell$.
Окружности $\omega_{1}$ и~$\omega_{2}$ с~центрами $O_{1}$ и~$O_{2}$
проходят через точки $A$ и~$B$,
а~окружности $\omega'_{1}$ и~$\omega'_{2}$ с~центрами $O_{1}'$ и~$O_{2}'$
проходят через точки $C$ и~$D$.
Пусть $\omega_{1} \perp \omega'_{1}$ и~$\omega_{2} \perp \omega'_{2}$.
Докажите, что прямые $O_{1}O'_{2}$, $O_{2}O'_{1}$ и~$\ell$ пересекаются
в~одной точке.

\item
Диагонали вписанного четырехугольника $ABCD$ пересекаются в~точке~$P$.
Окружность~$\Gamma$ касается продолжений $AB$, $BC$, $AD$, $DC$
в~точках $X$, $Y$, $Z$, $T$ соответственно.
Окружность~$\Omega$ проходит через $A$ и~$B$, и~касается внешне
окружности~$\Gamma$ в~точке~$S$.
Докажите, что $SP \perp ST$.

\item
На~плоскости даны $n$ различных точек и~круг радиуса~$r$ с~центром в~точке~$O$.
Хотя~бы одна точка лежит внутри круга.
На~каждом шаге мы передвигаем $O$ в~центр масс точек, попавших внутрь круга.
Докажите, что процесс стабилизируется.

\item
В~остроугольном треугольнике $ABC$ высоты $BF$ и~$CE$ пересекаются
в~ортоцентре~$H$, а~точка~$M$~--- середина стороны~$BC$.
Пусть точка~$X$ на~прямой~$EF$ такова, что $\angle XMH = \angle HAM$, причем
точки $A$ и~$X$ лежат по~разные стороны от~$MH$.
Докажите, что $AH$ делит $MX$ пополам.

\end{problems}

