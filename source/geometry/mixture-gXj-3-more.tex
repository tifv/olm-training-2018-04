% $date: 2018-04-06
% $timetable:
%   gXj:
%     2018-04-06:
%       2:

\worksheet*{Еще пять задач}

% $authors:
% - Фёдор Львович Бахарев

\begin{problems}

\item
На~стороне $BC$ параллелограмма $ABCD$ ($\angle A < 90^{\circ}$) отмечена
точка~$T$ так, что треугольник $ATD$~--- остроугольный.
Пусть $O_{1}$, $O_{2}$ и~$O_{3}$~--- центры описанных окружностей
треугольников $ABT$, $DAT$ и~$CDT$ соответственно.
Докажите, что ортоцентр треугольника $O_{1}O_{2}O_{3}$ лежит на~прямой~$AD$.

\item
Пусть $ABCD$~--- вписанный четырёхугольник, $O$~--- точка пересечения
диагоналей $AC$ и~$BD$.
Пусть окружности, описанные около треугольников $ABO$ и~$COD$, пересекаются
в~точке~$K$.
Точка~$L$ такова, что треугольник $BLC$ подобен треугольнику $AKD$.
Докажите, что если четырёхугольник $BLCK$ выпуклый, то~он он является
описанным.

\item
Пусть $A'$, $B'$ и~$C'$~--- точки касания вневписанных окружностей
с~соответствующими сторонами треугольника $ABC$.
Описанные окружности треугольников $A'B'C$, $AB'C'$ и~$A'BC'$ пересекают второй
раз описанную окружность треугольника $ABC$ в~точках $C_{1}$, $A_{1}$ и~$B_{1}$
соответственно.
Докажите, что треугольник $A_{1}B_{1}C_{1}$ подобен треугольнику, образованному
точками касания вписанной окружности треугольника с~его сторонами.

\item
Неравнобедренный треугольник $ABC$ вписан в~окружность~$\omega$.
Касательная к~этой окружности в~точке~$C$ пересекает прямую~$AB$ в~точке~$D$.
Пусть $I$~--- центр вписанной окружности, треугольника $ABC$.
Прямые $AI$ и~$BI$ пересекают биссектрису угла $CDB$ в~точках $Q$ и~$P$
соответственно.
Пусть $M$~--- середина отрезка~$PQ$.
Докажите, что прямая~$MI$ проходит через середину дуги~$ACB$
окружности~$\omega$.

\item
Треугольник $ABC$ ($AB > BC$) вписан в~окружность~$\Omega$.
На~сторонах $AB$ и~$BC$ выбраны точки $M$ и~$N$ соответственно так, что
$AM = CN$.
Прямые $MN$ и~$AC$ пересекаются в~точке~$K$.
Пусть $P$~--- центр вписанной окружности треугольника $AMK$, а~$Q$~--- центр
вневписанной окружности треугольника $CNK$, касающейся стороны~$CN$.
Докажите, что середина дуги~$ABC$ окружности~$\Omega$ равноудалена
от~точек~$P$ и~$Q$.

\end{problems}

