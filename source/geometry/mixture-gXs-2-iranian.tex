% $date: 2018-04-04
% $timetable:
%   gXs:
%     2018-04-04:
%       3:

\worksheet*{Иранская геометрия}

% $authors:
% - Фёдор Львович Бахарев

\begin{problems}

\item
Точки $E$ и~$F$ расположены соответственно на~сторонах $AB$ и~$AC$
треугольника $ABC$ на~одинаковом расстоянии от~середины стороны~$BC$.
Пусть $P$~--- вторая точка пересечения окружностей, описанных около
треугольников $ABC$ и~$AEF$.
Касательные в~точках $E$ и~$F$ к~окружности, описанной около треугольника $AEF$
пересекаются в~точке~$K$.
Докажите, что $\angle KPA = 90^{\circ}$.

\item
Дан треугольник $ABC$ с~ортоцентром~$H$ и~центром описанной окружности~$O$.
Пусть $R$~--- радиус описанной окружности треугольника $ABC$.
Определим точки $A'$, $B'$ и~$C'$ на~лучах $AH$, $BH$ и~$CH$ равенствами
$AH \cdot AA' = BH \cdot BB' = CH \cdot CC' = R^2$.
Докажите, что $O$~--- центр вписанной окружности треугольника $A'B'C'$.

\item
Прямая, проходящая через центр вписанной окружности~$I$ треугольника $ABC$
и~перпендикулярная $AI$, пересекает отрезки $AB$ и~$AC$ в~точках $B'$ и~$C'$
соответственно.
Точки $B''$ и~$C''$ выбраны на~лучах $BC$ и~$CB$ соответственно так, что
$BB'' = BA$ и~$CC'' = CA$.
Окружности, описанные около треугольников $AB'B''$ и~$AC'C''$, вторично
пересекаются в~точке~$T$.
Докажите, что центр описанной окружности треугольника $AIT$ лежит
на~прямой~$BC$.
% https://artofproblemsolving.com/community/c6h585433p3462669

\item
На~стороне~$BC$ треугольника $ABC$ выбрана точка~$D$.
Точки $I$, $I_{1}$ и~$I_{2}$~--- центры вписанных окружностей
треугольников $ABC$, $ABD$ и~$ACD$ соответственно.
Точки $M$ и~$N$~--- вторые точки пересечения окружностей, описанных около
треугольников $IAI_{1}$ и~$IAI_{2}$, с~окружностью, описанной около
треугольника $ABC$.
Докажите, что прямая~$MN$ проходит через фиксированную точку плоскости,
независящую от~положения точки~$D$
% https://artofproblemsolving.com/community/c6h619483p3697748

\item
Вписанная окружность треугольника $ABC$ с~центром, как обычно, в~точке~$I$
касается стороны~$BC$ в~точке~$D$.
Точка~$X$ на~дуге~$BC$ описанной около треугольника $ABC$ окружности такова,
что середина отрезка соединяющего проекции $E$, $F$ точки~$X$ на~$BI$ и~$CI$,
равноудалена от~вершин $B$ и~$C$.
Докажите, что $\angle BAD = \angle CAX$.

\item
Треугольник $ABC$ вписан в~окружность~$\omega$ с~центром~$O$.
Прямая~$AO$ пересекает $\omega$ вторично в~точке~$A'$.
Серединный перпендикуляр к~$OA'$ пересекает $BC$ в~точке~$P_{A}$.
Точки $P_{B}$ и~$P_{C}$ определяются аналогично.
Докажите, что
\\
\subproblem
точки $P_{A}$, $P_{B}$, $P_{C}$ коллинеарны;
\\
\subproblem
Расстояние от~точки~$O$ до~прямой $P_{A}P_{B}P_{C}$ равно $\frac{R}{2}$, где
$R$ радиус окружности~$\omega$.

\item
Точки $A$, $B$, $C$ и~$D$ лежат на~прямой~$\ell$ в~указанном порядке.
Дуги $\gamma_{1}$ и~$\gamma_{2}$, лежащие по~одну сторону от~$\ell$, проходят
через точки $A$ и~$B$.
Дуги $\gamma_{3}$ и~$\gamma_{4}$ проходят через $C$ и~$D$.
Оказалось, что $\gamma_{1}$ касается $\gamma_{3}$, а~$\gamma_{2}$ касается
$\gamma_{4}$.
Докажите, что общая внешняя касательная к~$\gamma_{2}$ и~$\gamma_{3}$ и~общая
внешняя касательная к~$\gamma_{1}$ и~$\gamma_{4}$ пересекаются
на~прямой~$\ell$.

\item
Четырехугольник $ABCD$ вписан в~окружность~$\omega$.
Пусть $I_{1}$, $I_{2}$ и~$r_{1}$, $r_{2}$ центры и~радиусы вписанных
окружностей $ACD$ и~$ABC$ соответственно.
Предположим, что $r_{1} = r_{2}$.
Окружность~$\omega'$ касается $AB$, $AD$ и~$\omega$ в~точке~$T$.
Касательные в~точках $A$, $T$ к~$\omega$ пересекаются в~точке~$K$.
Докажите, что $I_{1}$, $I_{2}$ и~$K$ лежат на~одной прямой.

\end{problems}

