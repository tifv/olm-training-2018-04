% $date: 2018-04-13
% $timetable:
%   gXj:
%     2018-04-13:
%       1:
%   gXs:
%     2018-04-13:
%       1:

\worksheet*{Тренировочная олимпиада~--- 2}

%% $authors:
%% - Андрей Юрьевич Кушнир
%% - Артемий Алексеевич Соколов

\begin{problems}

\item
Какое наименьшее количество клеток надо отметить на~доске $8 \times 8$, чтобы
среди любых пяти клеток, идущих подряд по~вертикали, горизонтали или диагонали,
была хотя~бы одна отмеченная?

\item
Про вещественные числа $a$, $b$, $c$ известно, что
\[
    \frac{a}{b - c} + \frac{b}{c - a} + \frac{c}{a - b} = 0
\, . \]
Докажите, что
\[
    \frac{a}{(b - c)^2} + \frac{b}{(c - a)^2} + \frac{c}{(a - b)^2} = 0
\, . \]

%\item
%$m$ и~$n$~--- натуральные числа, причем $1 < m < n - 1$.
%В~компании из~$n$ человек некоторые знакомы друг с~другом, причем среди любых
%$m$ человек из~этой компании одно и~тоже количество пар знакомых.
%Сколько пар знакомых может быть среди всех $n$ человек?

\item
Город Снежинск представляет собой квадрат со~стороной $100 n$ метров, разбитый
прямыми улицами на~$n^2$ одинаковых кварталов
($n$~--- четное натуральное число, большее $2$).
На~каждой из~$2 n + 2$ улиц, идущих по~сторонам кварталов от~края до~края
города, введено одностороннее движение.
На~соседних параллельных улицах движение направлено в~разные стороны.
В~одном из~углов города находится автопарк;
обе выходящие из~этого угла улицы направлены от~автопарка.
Снегоуборочная машина выезжает из~автопарка и~начинает убирать снег.
Чтобы не~портить дорожное покрытие, по~уже убранным участкам
(кроме перекрестков) машина не~ездит.
В~конце смены машина должна прибыть в~противоположный угол города.
Какое наибольшее расстояние она может проехать?

\item
Дан треугольник $ABC$.
Точки $D$ и~$E$ на~прямой~$AB$ (порядок точек $D {-} A {-} B {-} E$) таковы,
что $AD = AC$ и~$BE = BC$.
Биссектрисы углов $A$ и~$B$ пересекают стороны в~точках $P$ и~$Q$, а~описанную
окружность треугольника $ABC$ в~точках $M$ и~$N$ соответственно.
Прямые, соединяющие $A$ с~центром описанной окружности треугольника $BME$ и~$B$
с~центром описанной окружности треугольника $AND$, пересекаются в~точке~$X$.
Докажите, что $CX \perp PQ$.

%\itemy{2'}
%Дана некоторая перестановка $a_{1}$, $a_{2}$, \ldots, $a_{100}$
%чисел $1$, $2$, \ldots, $100$.
%Для всех натуральных $i \in [1; 100]$ определим $S_{i}$:
%\[
%    S_{1} = a_{1}
%\, , \;
%    S_{2} = a_{1} + a_{2}
%\, , \; \ldots, \;
%    S_{100} = a_{1} + a_{2} + \ldots + a_{100}
%\, . \]
%Какое наибольшее количество точных квадратов может быть среди всех $S_{i}$?

\end{problems}

