% $date: 2018-04-04
% $timetable:
%   gXs:
%     2018-04-04:
%       1:

\worksheet*{Тренировочная олимпиада~--- 1}

%% $authors:
%% - Андрей Юрьевич Кушнир
%% - Артемий Алексеевич Соколов

\begin{problems}

\item
Даны вещественные числа $x_{1}$, $x_{2}$, \ldots, $x_{n}$ ($n \geq 3$).
Известно, что \[
    x_{1} + x_{2} + \ldots + x_{n} = 0
\quad\text{и}\quad
    x_{1}^2 + x_{2}^2 + \ldots + x_{n}^2 = 1
\, . \]
Докажите, что существуют такие $i, j \in \{1, 2, \ldots,  n\}$, что
$x_{i} \cdot x_{j} \leq -\frac{1}{n}$.

\item
Дан параллелепипед $ABCDA_{1}B_{1}C_{1}D_{1}$.
Сфера~$S$ с~центром на~диагонали~$AC_{1}$ пересекает
ребра $AB$, $AD$, $AA_{1}$ в~точках $K$, $L$, $M$ соответственно,
а~ребра $C_{1}D_{1}$, $C_{1}B_{1}$, $C_{1}C$ в~точках $K_{1}$, $L_{1}$, $M_{1}$
соответственно.
Оказалось, что плоскости $KLM$ и~$K_{1}L_{1}M_{1}$ параллельны,
но~треугольники $KLM$ и~$K_{1}L_{1}M_{1}$ не~равны.
Докажите, что диагональЁ$AC_{1}$ образует равные углы с~ребрами
$AB$, $AD$, $AA_{1}$.

\item
ПоследовательностьЁ$a_n$ натуральных чисел называется \emph{фибоначчиевой,}
если для всех $n \geq 1$ выполнено $a_{n + 2} = a_{n + 1} + a_n$.
Можно~ли множество натуральных чисел представить в~виде объединения
(не~обязательно конечного) семейства попарно непересекающихся фибоначчиевых
последовательностей?

\item
Дано натуральное число~$k$.
На~бесконечной клетчатой плоскости каждая клетка является суверенным
государством, а~на~каждом ребре стоит таможня, взимающая натуральное число
талеров в~качестве взятки за~ее пересечение (в~обоих направлениях~---
одинаковое, но, возможно, различное для разных границ).
Докажите, что существует такой замкнутый маршрут, не~заходящий ни~в~какую
клетку дважды, что суммарная взятка на~нем кратна $k$.

\end{problems}

