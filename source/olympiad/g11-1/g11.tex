% $date: 2018-04-04
% $timetable:
%   g11:
%     2018-04-04:
%       1:

\worksheet*{Тренировочная олимпиада~--- 1}

%% $authors:
%% - Андрей Юрьевич Кушнир
%% - Артемий Алексеевич Соколов

\begin{problems}

\item
Даны вещественные числа $x_{1}$, $x_{2}$, \ldots, $x_{n}$ ($n \geq 3$).
Известно, что \[
    x_{1} + x_{2} + \ldots + x_{n} = 0
\quad\text{и}\quad
    x_{1}^2 + x_{2}^2 + \ldots + x_{n}^2 = 1
\, . \]
Докажите, что существуют такие $i, j \in \{1, 2, \ldots,  n\}$, что
$x_{i} \cdot x_{j} \leq -\frac{1}{n}$.

\item
Сфера, вписанная в~тетраэдр, касается одной из~его граней в~точке пересечения
биссектрис, другой~--- в~точке пересечения высот, третьей~--- в~точке
пересечения медиан.
Докажите, что тетраэдр правильный.

\item
Дана последовательность $\{a_{n}\}$ натуральных чисел, в~которой каждое
натуральное число встречается ровно один раз.
Известно, что для любых различных натуральных чисел $n$ и~$m$ верно неравенство
\[
    \frac{1}{2000} < \frac{|a_{n} - a_{m}|}{|n - m|} < 2000
\, . \]
Докажите, что при всех натуральных~$k$ выполнено $|a_{k} - k| < 2\,000\,000$.

\item
Дано натуральное число~$k$.
На~бесконечной клетчатой плоскости каждая клетка является суверенным
государством, а~на~каждом ребре стоит таможня, взимающая натуральное число
талеров в~качестве взятки за~ее пересечение (в~обоих направлениях~---
одинаковое, но, возможно, различное для разных границ).
Докажите, что существует такой замкнутый маршрут, не~заходящий ни~в~какую
клетку дважды, что суммарная взятка на~нем кратна $k$.

\end{problems}

