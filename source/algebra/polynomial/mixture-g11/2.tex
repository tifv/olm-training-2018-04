% $date: 2018-04-12
% $timetable:
%   g11:
%     2018-04-12:
%       2:

\worksheet*{Многочлены. Добавка}

% $authors:
% - Леонид Андреевич Попов

\begin{problems}

\item
Последовательность $a_{n}$ и~$b_{n}$ заданы условиями
\[
    a_{1} = 1
\, , \;
    b_{1} = 2
\, , \;
    a_{n+1} = \frac{1 + a_{n} + a_{n} b_{n}}{b_{n}}
\, , \;
    b_{n+1} = \frac{1 + b_{n} + a_{n} b_{n}}{a_{n}}
\, . \]
Докажите, что $a_{2018} < 5$.

\item
Даны положительные рациональные числа $a$, $b$.
Один из~корней трёхчлена $x^2 - a x + b$ является рациональным числом,
в~несократимой записи имеющим вид $m / n$.
Докажите, что знаменатель хотя~бы одного из~чисел $a$ и~$b$ (в~несократимой
записи) не~меньше $n^{2/3}$.

\item
Известно, что многочлен $(x + 1)^n - 1$ делится на~некоторый многочлен
\(
    P(x) = x^k + c_{k-1} x^{k-1} + c_{k-2} x^{k-2} + \ldots + c_{1}x + c_{0}
\)
чётной степени~$k$, у~которого все коэффициенты~--- целые нечётные числа.
Докажите, что $n$ делится на~$k + 1$.
% http://www.problems.ru/view_problem_details_new.php?id=109844

\end{problems}

