% $date: 2018-04-02
% $timetable:
%   g8:
%     2018-04-02:
%       2:

\worksheet*{Трёхчлен и~неравенства}

% $authors:
% - Александр Савельевич Штерн

\begingroup
    \def\abs#1{\lvert #1 \rvert}%

Неположительный дискриминант~--- условие знакопостоянства квадратного
трехчлена.

\claim{Неравенство Коши-Буняковского-Шварца}
\[
    (a_{1} b_{1} + a_{2} b_{2} + \ldots + a_{n} b_{n})^2
\leq
    (a_{1}^2 + a_{2}^2 + \ldots + a_{n}^2)
    \cdot
    (b_{1}^2 + b_{2}^2 + \ldots + b_{n}^2)
\, . \]

Доказывается рассмотрением трехчлена
\begin{align*}
    f(x)
& =
    (a_{1}^2 + a_{2}^2 + \ldots + a_{n}^2) x^2
    +
    2 (a_{1} b_{1} + a_{2} b_{2} + \ldots + a_{n} b_{n}) x
    +
    (b_{1}^2 + b_{2}^2 + \ldots + b_{n}^2)
= \\ & =
    (a_{1} x + b_{1})^2 + (a_{2} x + b_{2})^2 + \ldots + (a_{n} x + b_{n})^2
\end{align*}

Следствия:
\begin{exercises}
\item
Неравенство между средним арифметическим и~средним квадратичным:
$b_{1} = b_{2} =  \ldots = b_{n} = 1$.
\item
Неравенство между средним арифметическим и~средним гармоническим:
$a_{i} = \sqrt{x_{i}}$, $b_{i} = 1 / \sqrt{x_{i}}$.
\end{exercises}


\subsubsection*{Задачи для самостоятельного решения}

\begin{problems}

\item
Для положительных чисел $a$ и~$b$ выполнено неравенство $a^2 + b^2 > a + b$.
Докажите, что для этих чисел выполнено неравенство $a^3 + b^3 > a^2 + b^2$.

\item
Докажите, что для всех допустимых значений неизвестного выполнено неравенство
\[
    \sqrt{x + 1} + \sqrt{2 x - 3} + \sqrt{50 - 3 x} \leq 12
\, . \]

\item
Докажите, что для любых трех чисел, сумма квадратов которых равна $2$,
выполнено неравенство $a + b + c \leq a b c + 2$.

\item
Решите систему уравнений
$x^3 + y^2 = 2$, $x^2 + x y - y = 0$.

\item
Доказать, что для любых чисел $a$, $b$ справедливо неравенство
$a^2 + a b + b^2 \geq 3 (a + b - 1)$.

\item
Три различных числа таковы, что при любой их расстановке на~места коэффициентов
квадратного трехчлена будет получаться трехчлен, имеющий целый корень.
Докажите, что этот корень равен 1.

\item
Все значения квадратного трехчлена $a x^2 + b x + c$ на~отрезке $[0; 1]$
по~модулю не~превосходят $1$.
Какое наибольшее значение при этом может иметь величина
$\abs{a} + \abs{b} + \abs{c}$?

\item
Числа $a$ и~$b$ таковы, что каждый из~двух квадратных трехчленов
$x^2 + a x + b$ и~$x^2 + b x + a$ имеет по~два различных корня, а~произведение
этих трехчленов имеет ровно три различных корня.
Найдите все возможные значения суммы этих трех корней.
% Всерос 2014-15, 9.1

\item
Существуют~ли три квадратных трехчлена такие, что каждый из~них имеет корень,
а~сумма любых двух трехчленов не~имеет корней?

\item
Даны три различных числа $a$, $b$, $c$.
Докажите, что хотя~бы два из~уравнений
$(x - a) (x - b) = (x - c)$, $(x - a) (x - c) = (x - b)$,
$(x - c) (x - b) = (x - a)$ имеют корни.
% Всерос 2012-13, 9.1

\end{problems}

\endgroup % \def\abs

