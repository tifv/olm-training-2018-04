% $date: 2018-04-02
% $timetable:
%   gXj:
%     2018-04-02:
%       3:

\worksheet*{Диофантовы уравнения}

% $authors:
% - Александр Савельевич Штерн

\begin{problems}

\item
Докажите, что при любом натуральном $n$ уравнение
$y^2 = x_{1}^2 + x_{2}^2 + \ldots + x_{n}^2$
имеет решение в~натуральных числах.

\item \emph{(Замечание к~теории уравнений Маркова)}
Докажите, что пара чисел $(x, y)$ является решением диофантова уравнения
$x^2 + y^2 + 1 = 3 x y$ тогда и~только тогда, когда эти числа~--- числа
Фибоначчи с~последовательными нечётными номерами.

\item
Докажите, что уравнение $x^2 + y^3 = z^5$ имеет бесконечно много решений
в~натуральных числах.

\item
Решите в~натуральных числах уравнение $(x^2 - y^2)^2 = 1 + 16 y$.

\item
Решите в~натуральных числах уравнение $(1 + n^k)^l = 1 + n^m$, где $l > 1$.

\item
Натуральные числа $a$, $b$ выбраны так, что $(4 a^2 - 1)^2$ делится
на~$4 a b - 1$.
Докажите, что $a = b$.

\item
Докажите, что существует бесконечно много троек различных натуральных
чисел $a$, $b$, $c$ таких что у~чисел $a^2 + 1$, $b^2 + 1$ и~$c^2 + 1$
совпадают наибольшие простые делители.

\item
Докажите, что множество простых чисел вида $6 k + 1$ бесконечно.

\item
Натуральные числа $x$ и~$y$ таковы, что $2 x^2 - 1 = y^{15}$.
Докажите, что если $x > 1$, то~$x$ делится на~$5$.

\end{problems}

